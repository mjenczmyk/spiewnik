\tytul{Panna Kminkowa}{sł. muz. Jerzy Reiser}{Browar Żywiec}
\begin{textn}
    Obróciła się wiosna na pięcie\\
    Wyminęła się z latem we drzwiach\\
    Wpatrywałaś się w okno zawzięcie\\
    Może wyjdzie kominiarz na dach
        
    Potem jechał gdzieś pociąg spóźniony\\
    Ktoś trzy asy wyłożył na stół\\
    A Ty kwiaty wkładałaś w wazony\\
    I słyszałaś z daleka stuk kół
        
    A tymczasem rozgościł się lipiec\\
    Na wesele zaprosił ze stu\\
    Pasikonik pożyczył mu skrzypiec\\
    I w tym graniu tak biegłaś bez tchu
        
    \vin Panno kminkowa, panno lipcowa\\
    \vin Twoje są łąki, Twoje skowronki wszystkich łąk\\
    \vin Idziesz polami, w palcach łodyżkę masz\\
    \vin I gryziesz kminek, czarny przecinek nasz
        
    \vin Panna kminkowa, panna lipcowa\\
    \vin Śmiać się gotowa letnia królowa kwietnych łąk\\
    \vin Idzie polami, w palcach łodyżkę ma\\
    \vin I gryzie kminek, czarny przecinek dnia
        
    Noce śniły, budziły się ranki\\
    Koniczyny poczwórny był liść\\
    Rwałaś groszek zielony przed gankiem\\
    Kiedy przyszło polami nam iść
        
    Trwało lato i trwała muzyka\\
    Rumieniłaś się wiśnią z mych ust\\
    Las na wzgórzu horyzont zamykał\\
    A Ty biegłaś ku niemu bez tchu
        
    \vin Panno kminkowa, panno lipcowa...
        
    A gdy lipiec się w drogę spakował\\
    Oddał skrzypce i poszedł gdzieś w świat\\
    Ty zostałaś mi, panno kminkowa\\
    I łodyżki rzucane na wiatr
        
    Jesień deszczem Twe łąki przemoczy\\
    Zima śniegiem okryje na mróz\\
    A gdy lipiec zaglądnie Ci w oczy\\
    Znowu będziesz tak biegła bez tchu
        
    \vin Panno kminkowa, panno lipcowa...
\end{textn}
\begin{chordw}
    G D G\\
    C C^{7+} D\\
    G D G\\
    C C^{7+} D G H^7 e
        
    e H^7 e\\
    C C^{97+} C^6 D\\
    G D G\\
    a C^{97+} C D
        
    G D G\\
    C C^{97+} D\\
    e H^7 e\\
    C D
        
    G D e D\\
    G G D D\\
    e C D G C\\
    G C D G G
        
    G D e D\\
    G G D D\\
    e C D G C\\
    G C D G G
        
    G D G\\
    C C^{7+} D\\
    G D G\\
    C C^{7+} D G H^7 e
        
    e H^7 e\\
    C C^{97+} C^6 D\\
    G D G\\
    a C^{97+} C D\\    
    \hfill\break
    
    G D G\\
    C C^{7+} D\\
    G D G\\
    C C^{7+} D G H^7 e
        
    e H^7 e\\
    C C^{97+} C^6 D\\
    G D G\\
    a C^{97+} C D
\end{chordw}
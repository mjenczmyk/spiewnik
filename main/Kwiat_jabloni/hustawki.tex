\tytul{Huśtawki}{sł. muz. Jacek Kleyff}{Kwiat Jabłoni}
\begin{text}
    A czy przyroda kolebka\\ C
    Myślała kiedyś dokładnie\\ G
    Po co jej wielkie mamuty\\ a
    Ani wygląda to ładnie\\ F
    Ani z nich skóra na buty G

    Nie ma co pytać koledzy\\
    Robiła i tak jej wyszło\\
    Nikt nie wymyślał specjalnie\\
    Tego w czym żyć nam przyszło\\
    Uprzedzam o tym lojalnie\\

    Jeden jest rytm, jeden rytm\\
    Jeden jest węgiel i tlen\\
    Zwykłą losu koleją\\
    Praca posiłek i sen\\
    Praca posiłek i sen

    Jeden przypada na dzień\\
    Świt jeden i jeden zmrok\\
    Pierwsi się łudzą nadzieją\\
    A drudzy równają krok\\
    A drudzy równają krok\\

    Nie skacz tak zaraz na szyny\\
    Jeszcze nie o tę grasz stawkę\\
    W wesołym miasteczku dziewczyny\\
    Chcą z tobą iść na huśtawkę...\\
    Lepiej ci będzie nimi

    Pachnie tak mocno siano\\
    Kwiaty się gną od motyli\\
    Jeździ słońce po niebie\\
    Światło ucieka, ślad myli\\
    Miasteczko czeka na ciebie

    Jeden jest rytm, jeden rytm\\
    Jeden jest wydech i wdech\\
    Nasyć się równym oddechem\\
    Nasyć się dzisiaj za trzech\\
    Nasyć się dzisiaj za trzech

    Raz tylko dany ten czas\\
    Ani on twój ani czyj\\
    Z czasem się wszystko ustoi\\
    Żyj na huśtawce żyj\\
    Żyj na huśtawce żyj

    Jeden jest rytm, jeden rytm\\
    Jeden jest węgiel i tlen\\
    Zwykłą losu koleją\\
    Praca posiłek i sen\\
    Praca posiłek i sen    

    Raz tylko dany ten czas\\
    Ani on twój ani czyj\\
    Z czasem się wszystko ustoi\\
    Żyj na huśtawce żyj
\end{text}
\begin{chord}
    C\\
    G\\
    a\\
    F\\
    G
\end{chord}
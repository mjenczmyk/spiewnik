\tytul{Motyw pociągu}{sł.A.Kulawczuk, muz. B. Adamczak}{Na Bani}
\begin{text}
Przewraca się nocą, coś znów jej się jawi\\
stare żółte kartki ogląda pod światłem\\
szuka śladów palców, co dla niej pisały\\
wiersze miłosne - tęsknoty liryczne\\
w pragnieniach żarliwych przestrogi niedbałe

\vin Bo potem to tylko kolejnym pociągom\\
\vin uśmiechy staniałe, rutyna pożegnań\\
\vin swobodny ruch dłoni palcami, co nie drżą\\
\vin powroty wieczorne i znowu poduszce\\
\vin nad głową niespokojną serce z żalu pęka

Starannie układa w pamięci wzruszenia\\
papieros odtwarza ulotnym obłokiem\\
te dłonie stroskane, co nad nią w jej snach\\
mgły nad wodami - i za jego zdrowie\\
w następną rocznicę haust dymu do dna
\end{text}
\begin{chord}
    d B\\
    F $\mathrm{A^7}$\\
    d B\\
    F $\mathrm{A^7}$\\
    B d $\mathrm{A^7}$\\
    D $\mathrm{A^4}$ C G\\
    D $\mathrm{A^4}$\\
    C G\\
    D $\mathrm{A^4}$\\
    C $\mathrm{A^4}$\\
    \small{$\mathrm{h^7}$ $\mathrm{C^9}$ $\mathrm{A^4}$CGD}\\
\end{chord}

\tytul{Rapapara}{}{Łydka Grubasa}
\begin{text}
    On był samotny, jej było źle\\
    Gdzieś w internecie poznali się\\
    On się zakochał ze samych zdjęć\\
    Bo tam rusałka, dziewczę na pięć\\
    Szczęka mu spadła aż pod sam stół\\
    Dał jej komentarz dziesięć i pół\\
    A kiedy w końcu spotkali się\\
    On jej nie poznał dlatego, że...

    \vin Rapapara, rapapara miała ryja jak kopara\\
    \vin Rapapara, rapapara miała ryja jak kopara

    On chciał zakochać się z całych sił\\
    Lecz ciągle widział ten wielki ryj\\
    W łóżku i w pracy, noce i dnie\\
    Z hipopotamem kojarzył się

    \vin Rapapara, rapapara miała ryja jak kopara\\
    \vin Rapapara, rapapara miała ryja jak kopara

    A w końcu przyszedł zimowy czas\\
    Śniegu nasypało aż po pas\\
    Gdy on do pracy wyruszyć chciał\\
    Ujrzał, że w śniegu ugrzązł mu star\\
    Płacząc przeklinał parszywy los\\
    Wtedy "pomogę" usłyszał głos\\
    I kiedy w starze zarzucał bieg\\
    To ona ryjem spychała śnieg

    \vin Rapapara, rapapara i tym ryjem jak kopara\\
    \vin Rapapara, rapapara odkopała chłopu stara

    Przyznaj się teraz, przyznaj się sam\\
    Śmiałeś się z ryja, śmiałeś jak cham\\
    Brałeś do ręki sękaty kij\\
    Plułeś, i szczułeś ten wielki ryj\\
    (Lecz) Karty rozdaje parszywy los\\
    Mniej bywa cenny jak złota stos\\
    A więc nie śmiejcie się z cudzych wad\\
    Bo one mogą zbawić wasz świat

    \vin Rapapara, rapapara nawet morda jak kopara\\
    \vin Rapapara, rapapara zasługuje na browara!	
\end{text}
\begin{chord}
    a F\\
    C G\\
    a F\\
    C G\\
    a F\\
    C G\\
    a F\\
    C G

    a F C G\\
    a F C G
\end{chord}
